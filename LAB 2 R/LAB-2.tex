% Options for packages loaded elsewhere
\PassOptionsToPackage{unicode}{hyperref}
\PassOptionsToPackage{hyphens}{url}
%
\documentclass[
]{article}
\usepackage{amsmath,amssymb}
\usepackage{lmodern}
\usepackage{iftex}
\ifPDFTeX
  \usepackage[T1]{fontenc}
  \usepackage[utf8]{inputenc}
  \usepackage{textcomp} % provide euro and other symbols
\else % if luatex or xetex
  \usepackage{unicode-math}
  \defaultfontfeatures{Scale=MatchLowercase}
  \defaultfontfeatures[\rmfamily]{Ligatures=TeX,Scale=1}
\fi
% Use upquote if available, for straight quotes in verbatim environments
\IfFileExists{upquote.sty}{\usepackage{upquote}}{}
\IfFileExists{microtype.sty}{% use microtype if available
  \usepackage[]{microtype}
  \UseMicrotypeSet[protrusion]{basicmath} % disable protrusion for tt fonts
}{}
\makeatletter
\@ifundefined{KOMAClassName}{% if non-KOMA class
  \IfFileExists{parskip.sty}{%
    \usepackage{parskip}
  }{% else
    \setlength{\parindent}{0pt}
    \setlength{\parskip}{6pt plus 2pt minus 1pt}}
}{% if KOMA class
  \KOMAoptions{parskip=half}}
\makeatother
\usepackage{xcolor}
\usepackage[margin=1in]{geometry}
\usepackage{color}
\usepackage{fancyvrb}
\newcommand{\VerbBar}{|}
\newcommand{\VERB}{\Verb[commandchars=\\\{\}]}
\DefineVerbatimEnvironment{Highlighting}{Verbatim}{commandchars=\\\{\}}
% Add ',fontsize=\small' for more characters per line
\usepackage{framed}
\definecolor{shadecolor}{RGB}{248,248,248}
\newenvironment{Shaded}{\begin{snugshade}}{\end{snugshade}}
\newcommand{\AlertTok}[1]{\textcolor[rgb]{0.94,0.16,0.16}{#1}}
\newcommand{\AnnotationTok}[1]{\textcolor[rgb]{0.56,0.35,0.01}{\textbf{\textit{#1}}}}
\newcommand{\AttributeTok}[1]{\textcolor[rgb]{0.77,0.63,0.00}{#1}}
\newcommand{\BaseNTok}[1]{\textcolor[rgb]{0.00,0.00,0.81}{#1}}
\newcommand{\BuiltInTok}[1]{#1}
\newcommand{\CharTok}[1]{\textcolor[rgb]{0.31,0.60,0.02}{#1}}
\newcommand{\CommentTok}[1]{\textcolor[rgb]{0.56,0.35,0.01}{\textit{#1}}}
\newcommand{\CommentVarTok}[1]{\textcolor[rgb]{0.56,0.35,0.01}{\textbf{\textit{#1}}}}
\newcommand{\ConstantTok}[1]{\textcolor[rgb]{0.00,0.00,0.00}{#1}}
\newcommand{\ControlFlowTok}[1]{\textcolor[rgb]{0.13,0.29,0.53}{\textbf{#1}}}
\newcommand{\DataTypeTok}[1]{\textcolor[rgb]{0.13,0.29,0.53}{#1}}
\newcommand{\DecValTok}[1]{\textcolor[rgb]{0.00,0.00,0.81}{#1}}
\newcommand{\DocumentationTok}[1]{\textcolor[rgb]{0.56,0.35,0.01}{\textbf{\textit{#1}}}}
\newcommand{\ErrorTok}[1]{\textcolor[rgb]{0.64,0.00,0.00}{\textbf{#1}}}
\newcommand{\ExtensionTok}[1]{#1}
\newcommand{\FloatTok}[1]{\textcolor[rgb]{0.00,0.00,0.81}{#1}}
\newcommand{\FunctionTok}[1]{\textcolor[rgb]{0.00,0.00,0.00}{#1}}
\newcommand{\ImportTok}[1]{#1}
\newcommand{\InformationTok}[1]{\textcolor[rgb]{0.56,0.35,0.01}{\textbf{\textit{#1}}}}
\newcommand{\KeywordTok}[1]{\textcolor[rgb]{0.13,0.29,0.53}{\textbf{#1}}}
\newcommand{\NormalTok}[1]{#1}
\newcommand{\OperatorTok}[1]{\textcolor[rgb]{0.81,0.36,0.00}{\textbf{#1}}}
\newcommand{\OtherTok}[1]{\textcolor[rgb]{0.56,0.35,0.01}{#1}}
\newcommand{\PreprocessorTok}[1]{\textcolor[rgb]{0.56,0.35,0.01}{\textit{#1}}}
\newcommand{\RegionMarkerTok}[1]{#1}
\newcommand{\SpecialCharTok}[1]{\textcolor[rgb]{0.00,0.00,0.00}{#1}}
\newcommand{\SpecialStringTok}[1]{\textcolor[rgb]{0.31,0.60,0.02}{#1}}
\newcommand{\StringTok}[1]{\textcolor[rgb]{0.31,0.60,0.02}{#1}}
\newcommand{\VariableTok}[1]{\textcolor[rgb]{0.00,0.00,0.00}{#1}}
\newcommand{\VerbatimStringTok}[1]{\textcolor[rgb]{0.31,0.60,0.02}{#1}}
\newcommand{\WarningTok}[1]{\textcolor[rgb]{0.56,0.35,0.01}{\textbf{\textit{#1}}}}
\usepackage{graphicx}
\makeatletter
\def\maxwidth{\ifdim\Gin@nat@width>\linewidth\linewidth\else\Gin@nat@width\fi}
\def\maxheight{\ifdim\Gin@nat@height>\textheight\textheight\else\Gin@nat@height\fi}
\makeatother
% Scale images if necessary, so that they will not overflow the page
% margins by default, and it is still possible to overwrite the defaults
% using explicit options in \includegraphics[width, height, ...]{}
\setkeys{Gin}{width=\maxwidth,height=\maxheight,keepaspectratio}
% Set default figure placement to htbp
\makeatletter
\def\fps@figure{htbp}
\makeatother
\setlength{\emergencystretch}{3em} % prevent overfull lines
\providecommand{\tightlist}{%
  \setlength{\itemsep}{0pt}\setlength{\parskip}{0pt}}
\setcounter{secnumdepth}{-\maxdimen} % remove section numbering
\ifLuaTeX
  \usepackage{selnolig}  % disable illegal ligatures
\fi
\IfFileExists{bookmark.sty}{\usepackage{bookmark}}{\usepackage{hyperref}}
\IfFileExists{xurl.sty}{\usepackage{xurl}}{} % add URL line breaks if available
\urlstyle{same} % disable monospaced font for URLs
\hypersetup{
  pdftitle={R Notebook},
  hidelinks,
  pdfcreator={LaTeX via pandoc}}

\title{R Notebook}
\author{}
\date{\vspace{-2.5em}}

\begin{document}
\maketitle

This is an \href{http://rmarkdown.rstudio.com}{R Markdown} Notebook.
When you execute code within the notebook, the results appear beneath
the code.

Try executing this chunk by clicking the \emph{Run} button within the
chunk or by placing your cursor inside it and pressing
\emph{Ctrl+Shift+Enter}.

\begin{Shaded}
\begin{Highlighting}[]
\NormalTok{dat }\OtherTok{=} \FunctionTok{read.csv}\NormalTok{(}\StringTok{"D:/R LAB DAFE/datasets/Stock\_Bond.csv"}\NormalTok{)}
\FunctionTok{attach}\NormalTok{(dat)}
\FunctionTok{par}\NormalTok{(}\AttributeTok{mfcol =} \FunctionTok{c}\NormalTok{(}\DecValTok{2}\NormalTok{, }\DecValTok{1}\NormalTok{))}
\FunctionTok{plot}\NormalTok{(GM\_AC, }\AttributeTok{type =} \StringTok{"l"}\NormalTok{)}
\FunctionTok{plot}\NormalTok{(F\_AC, }\AttributeTok{type =} \StringTok{"l"}\NormalTok{)}
\end{Highlighting}
\end{Shaded}

\includegraphics{LAB-2_files/figure-latex/unnamed-chunk-1-1.pdf}

\begin{Shaded}
\begin{Highlighting}[]
\FunctionTok{plot}\NormalTok{(F\_AC, }\AttributeTok{type =} \StringTok{"l"}\NormalTok{)}
\end{Highlighting}
\end{Shaded}

\includegraphics{LAB-2_files/figure-latex/unnamed-chunk-2-1.pdf}

\begin{Shaded}
\begin{Highlighting}[]
\NormalTok{n }\OtherTok{=} \FunctionTok{dim}\NormalTok{(dat)[}\DecValTok{1}\NormalTok{]}
\NormalTok{GMReturn }\OtherTok{=}\NormalTok{ GM\_AC[}\SpecialCharTok{{-}}\DecValTok{1}\NormalTok{] }\SpecialCharTok{/}\NormalTok{ GM\_AC[}\SpecialCharTok{{-}}\NormalTok{n] }\SpecialCharTok{{-}} \DecValTok{1}
\NormalTok{FReturn }\OtherTok{=}\NormalTok{ F\_AC[}\SpecialCharTok{{-}}\DecValTok{1}\NormalTok{] }\SpecialCharTok{/}\NormalTok{ F\_AC[}\SpecialCharTok{{-}}\NormalTok{n] }\SpecialCharTok{{-}} \DecValTok{1}
\FunctionTok{par}\NormalTok{(}\AttributeTok{mfrow =} \FunctionTok{c}\NormalTok{(}\DecValTok{1}\NormalTok{, }\DecValTok{1}\NormalTok{))}
\FunctionTok{plot}\NormalTok{(GMReturn,FReturn)}
\end{Highlighting}
\end{Shaded}

\includegraphics{LAB-2_files/figure-latex/unnamed-chunk-3-1.pdf}

computing the log returns for GM and plotting the returns versus the log
returns

\begin{Shaded}
\begin{Highlighting}[]
\NormalTok{GMLogreturn }\OtherTok{=} \FunctionTok{log}\NormalTok{(GM\_AC[}\SpecialCharTok{{-}}\DecValTok{1}\NormalTok{]}\SpecialCharTok{/}\NormalTok{GM\_AC[}\SpecialCharTok{{-}}\NormalTok{n])}
\FunctionTok{plot}\NormalTok{(GMReturn, GMLogreturn)}
\end{Highlighting}
\end{Shaded}

\includegraphics{LAB-2_files/figure-latex/unnamed-chunk-4-1.pdf}

\begin{Shaded}
\begin{Highlighting}[]
\FunctionTok{cor}\NormalTok{(GMReturn, GMLogreturn)}
\end{Highlighting}
\end{Shaded}

\begin{verbatim}
## [1] 0.9995408
\end{verbatim}

Repeat Problem 1 with Microsoft (MSFT) and Merck (MRK)

\begin{Shaded}
\begin{Highlighting}[]
\NormalTok{MSFTreturn }\OtherTok{=}\NormalTok{ MSFT\_AC[}\SpecialCharTok{{-}}\DecValTok{1}\NormalTok{]}\SpecialCharTok{/}\NormalTok{MSFT\_AC[}\SpecialCharTok{{-}}\NormalTok{n] }\SpecialCharTok{{-}}\DecValTok{1}
\NormalTok{MRKreturn }\OtherTok{=}\NormalTok{ MRK\_AC[}\SpecialCharTok{{-}}\DecValTok{1}\NormalTok{]}\SpecialCharTok{/}\NormalTok{MRK\_AC[}\SpecialCharTok{{-}}\NormalTok{n] }\SpecialCharTok{{-}}\DecValTok{1}
\FunctionTok{plot}\NormalTok{(MSFTreturn, MRKreturn)}
\end{Highlighting}
\end{Shaded}

\includegraphics{LAB-2_files/figure-latex/unnamed-chunk-6-1.pdf}

\hypertarget{section}{%
\section{}\label{section}}

SIMULATIONS

\begin{Shaded}
\begin{Highlighting}[]
\NormalTok{niter }\OtherTok{=} \FloatTok{1e5} \CommentTok{\# number of iterations}
\NormalTok{below }\OtherTok{=} \FunctionTok{rep}\NormalTok{(}\DecValTok{0}\NormalTok{, niter) }\CommentTok{\# set up storage}
\FunctionTok{set.seed}\NormalTok{(}\DecValTok{2009}\NormalTok{)}
\ControlFlowTok{for}\NormalTok{ (i }\ControlFlowTok{in} \DecValTok{1}\SpecialCharTok{:}\NormalTok{niter)}
\NormalTok{\{}
\NormalTok{r }\OtherTok{=} \FunctionTok{rnorm}\NormalTok{(}\DecValTok{45}\NormalTok{, }\AttributeTok{mean =} \FloatTok{0.05}\SpecialCharTok{/}\DecValTok{253}\NormalTok{, }\AttributeTok{sd =} \FloatTok{0.23}\SpecialCharTok{/}\FunctionTok{sqrt}\NormalTok{(}\DecValTok{253}\NormalTok{)) }\CommentTok{\# generate random numbers}
\NormalTok{logPrice }\OtherTok{=} \FunctionTok{log}\NormalTok{(}\FloatTok{1e6}\NormalTok{) }\SpecialCharTok{+} \FunctionTok{cumsum}\NormalTok{(r)}
\NormalTok{minlogP }\OtherTok{=} \FunctionTok{min}\NormalTok{(logPrice) }\CommentTok{\# minimum price over next 45 days}
\NormalTok{below[i] }\OtherTok{=} \FunctionTok{as.numeric}\NormalTok{(minlogP }\SpecialCharTok{\textless{}} \FunctionTok{log}\NormalTok{(}\DecValTok{950000}\NormalTok{))}
\NormalTok{\}}
\FunctionTok{mean}\NormalTok{(below)}
\end{Highlighting}
\end{Shaded}

\begin{verbatim}
## [1] 0.50988
\end{verbatim}

On line 10, below{[}i{]} equals one if, for the ith simulation, the
minimum price over 45 days is less than 950,000. Therefore, on line 12,
mean(below) is the proportion of simulations where the minimum price is
less than 950,000.

\begin{Shaded}
\begin{Highlighting}[]
\FunctionTok{set.seed}\NormalTok{(}\DecValTok{2012}\NormalTok{)}
\NormalTok{n }\OtherTok{=} \DecValTok{253}
\FunctionTok{par}\NormalTok{(}\AttributeTok{mfrow=}\FunctionTok{c}\NormalTok{(}\DecValTok{3}\NormalTok{,}\DecValTok{3}\NormalTok{))}
\ControlFlowTok{for}\NormalTok{ (i }\ControlFlowTok{in}\NormalTok{ (}\DecValTok{1}\SpecialCharTok{:}\DecValTok{9}\NormalTok{))}
\NormalTok{\{}
\NormalTok{logr }\OtherTok{=} \FunctionTok{rnorm}\NormalTok{(n, }\FloatTok{0.05} \SpecialCharTok{/} \DecValTok{253}\NormalTok{, }\FloatTok{0.2} \SpecialCharTok{/} \FunctionTok{sqrt}\NormalTok{(}\DecValTok{253}\NormalTok{))}
\NormalTok{price }\OtherTok{=} \FunctionTok{c}\NormalTok{(}\DecValTok{120}\NormalTok{, }\DecValTok{120} \SpecialCharTok{*} \FunctionTok{exp}\NormalTok{(}\FunctionTok{cumsum}\NormalTok{(logr)))}
\FunctionTok{plot}\NormalTok{(price, }\AttributeTok{type =} \StringTok{"b"}\NormalTok{)}
\NormalTok{\}}
\end{Highlighting}
\end{Shaded}

\includegraphics{LAB-2_files/figure-latex/unnamed-chunk-8-1.pdf}

\begin{Shaded}
\begin{Highlighting}[]
\NormalTok{data}\OtherTok{=} \FunctionTok{read.csv}\NormalTok{(}\StringTok{"D:/R LAB DAFE/datasets/MCD\_PriceDaily.csv"}\NormalTok{)}
\FunctionTok{attach}\NormalTok{(data)}
\end{Highlighting}
\end{Shaded}

\begin{verbatim}
## The following object is masked from dat:
## 
##     Date
\end{verbatim}

\begin{Shaded}
\begin{Highlighting}[]
\FunctionTok{head}\NormalTok{(data)}
\end{Highlighting}
\end{Shaded}

\begin{verbatim}
##        Date  Open  High   Low Close   Volume Adj.Close
## 1  1/4/2010 62.63 63.07 62.31 62.78  5839300     53.99
## 2  1/5/2010 62.66 62.75 62.19 62.30  7099000     53.58
## 3  1/6/2010 62.20 62.41 61.06 61.45 10551300     52.85
## 4  1/7/2010 61.25 62.34 61.11 61.90  7517700     53.24
## 5  1/8/2010 62.27 62.41 61.60 61.84  6107300     53.19
## 6 1/11/2010 62.02 62.43 61.85 62.32  6081300     53.60
\end{verbatim}

\begin{Shaded}
\begin{Highlighting}[]
\NormalTok{adjPrice }\OtherTok{=}\NormalTok{ data[, }\DecValTok{7}\NormalTok{]}
\end{Highlighting}
\end{Shaded}

Compute the returns and log returns and plot them against each other. As
discussed in Sect. 2.1.3, does it seem reasonable that the two types of
daily returns are approximately equal?

\begin{Shaded}
\begin{Highlighting}[]
\NormalTok{len }\OtherTok{=} \FunctionTok{dim}\NormalTok{(data)[}\DecValTok{1}\NormalTok{]}
\NormalTok{return\_MCD }\OtherTok{=}\NormalTok{ Adj.Close[}\SpecialCharTok{{-}}\DecValTok{1}\NormalTok{]}\SpecialCharTok{/}\NormalTok{Adj.Close[}\SpecialCharTok{{-}}\NormalTok{len] }\SpecialCharTok{{-}}\DecValTok{1}
\NormalTok{Lreturn\_MCD }\OtherTok{=} \FunctionTok{log}\NormalTok{(Adj.Close[}\SpecialCharTok{{-}}\DecValTok{1}\NormalTok{]}\SpecialCharTok{/}\NormalTok{Adj.Close[}\SpecialCharTok{{-}}\NormalTok{len])}
\FunctionTok{plot}\NormalTok{(return\_MCD, Lreturn\_MCD, }\AttributeTok{type =} \StringTok{"b"}\NormalTok{)}
\end{Highlighting}
\end{Shaded}

\includegraphics{LAB-2_files/figure-latex/unnamed-chunk-10-1.pdf}

Compute the mean and standard deviation for both the returns and the log
returns. Comment on the similarities and differences you perceive in the
first two moments of each random variable. Does it seem reasonable that
they are the same?

\begin{Shaded}
\begin{Highlighting}[]
\FunctionTok{mean}\NormalTok{(return\_MCD)}
\end{Highlighting}
\end{Shaded}

\begin{verbatim}
## [1] 0.0005027479
\end{verbatim}

\begin{Shaded}
\begin{Highlighting}[]
\FunctionTok{sd}\NormalTok{(return\_MCD)}
\end{Highlighting}
\end{Shaded}

\begin{verbatim}
## [1] 0.008900319
\end{verbatim}

\begin{Shaded}
\begin{Highlighting}[]
\FunctionTok{mean}\NormalTok{(Lreturn\_MCD)}
\end{Highlighting}
\end{Shaded}

\begin{verbatim}
## [1] 0.0004630553
\end{verbatim}

\begin{Shaded}
\begin{Highlighting}[]
\FunctionTok{sd}\NormalTok{(Lreturn\_MCD)}
\end{Highlighting}
\end{Shaded}

\begin{verbatim}
## [1] 0.008901467
\end{verbatim}

\begin{Shaded}
\begin{Highlighting}[]
\FunctionTok{t.test}\NormalTok{(return\_MCD, Lreturn\_MCD, }\AttributeTok{paired=}\ConstantTok{TRUE}\NormalTok{)}
\end{Highlighting}
\end{Shaded}

\begin{verbatim}
## 
##  Paired t-test
## 
## data:  return_MCD and Lreturn_MCD
## t = 15.866, df = 1175, p-value < 2.2e-16
## alternative hypothesis: true mean difference is not equal to 0
## 95 percent confidence interval:
##  3.478409e-05 4.460108e-05
## sample estimates:
## mean difference 
##    3.969258e-05
\end{verbatim}

Assume that McDonald's log returns are normally distributed with mean
and standard deviation equal to their estimates and that you have been
made the following proposition by a friend: If at any point within the
next 20 trading days, the price of McDonald's falls below 85 dollars,
you will be paid \$100, but if it does not, you have to pay him \$1. The
current price of McDonald's is at the end of the sample data, \$93.07.
Are you willing to make the bet? (Use 10,000 iterations in your
simulation and use the command set.seed(2015) to ensure your results are
the same as the answer key)

\begin{Shaded}
\begin{Highlighting}[]
\FunctionTok{set.seed}\NormalTok{(}\DecValTok{2015}\NormalTok{)}
\NormalTok{n }\OtherTok{=} \FloatTok{1e4}
\NormalTok{c }\OtherTok{=} \FunctionTok{rep}\NormalTok{(}\DecValTok{0}\NormalTok{, n)}
\ControlFlowTok{for}\NormalTok{(i }\ControlFlowTok{in}\NormalTok{ (}\DecValTok{1}\SpecialCharTok{:}\NormalTok{n))\{}
\NormalTok{  LR }\OtherTok{=} \FunctionTok{rnorm}\NormalTok{(}\DecValTok{20}\NormalTok{, }\AttributeTok{mean =} \FunctionTok{mean}\NormalTok{(Lreturn\_MCD), }\AttributeTok{sd =} \FunctionTok{sd}\NormalTok{(Lreturn\_MCD))}
\NormalTok{  pr }\OtherTok{=} \FloatTok{93.07}\SpecialCharTok{*}\FunctionTok{exp}\NormalTok{(}\FunctionTok{cumsum}\NormalTok{(LR))}
  \ControlFlowTok{if}\NormalTok{(}\FunctionTok{min}\NormalTok{(pr) }\SpecialCharTok{\textless{}} \DecValTok{85}\NormalTok{)\{}
\NormalTok{    c[i] }\OtherTok{=} \DecValTok{100}
\NormalTok{  \}}
  \ControlFlowTok{else}\NormalTok{\{}
\NormalTok{    c[i] }\OtherTok{=} \SpecialCharTok{{-}}\DecValTok{1}
\NormalTok{  \}}
\NormalTok{\}}
\FunctionTok{mean}\NormalTok{(c)}
\end{Highlighting}
\end{Shaded}

\begin{verbatim}
## [1] 0.0605
\end{verbatim}

Add a new chunk by clicking the \emph{Insert Chunk} button on the
toolbar or by pressing \emph{Ctrl+Alt+I}.

When you save the notebook, an HTML file containing the code and output
will be saved alongside it (click the \emph{Preview} button or press
\emph{Ctrl+Shift+K} to preview the HTML file).

The preview shows you a rendered HTML copy of the contents of the
editor. Consequently, unlike \emph{Knit}, \emph{Preview} does not run
any R code chunks. Instead, the output of the chunk when it was last run
in the editor is displayed.

\end{document}
